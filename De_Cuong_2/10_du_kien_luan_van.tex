\section{Dự kiến nội dung của luận văn} % Section 10

Luận văn dự kiến được chia thành 5 chương chính như sau:

\begin{itemize}
    \item \textbf{Chương 1 – Giới thiệu đề tài}  
    \begin{itemize}
        \item Bối cảnh, lý do chọn đề tài  
        \item Mục tiêu, đối tượng, phạm vi nghiên cứu  
        \item Ý nghĩa khoa học và thực tiễn  
    \end{itemize}

    \item \textbf{Chương 2 – Cơ sở lý thuyết và tổng quan nghiên cứu}  
    \begin{itemize}
        \item Các kỹ thuật nhận diện biểu cảm khuôn mặt  
        \item Kiến trúc CNN nhẹ (đặc biệt là MobileNetV3)  
        \item Kỹ thuật tăng cường dữ liệu và các công trình liên quan  
    \end{itemize}

    \item \textbf{Chương 3 – Phương pháp đề xuất}  
    \begin{itemize}
        \item Chi tiết kỹ thuật tăng cường dữ liệu thích ứng  
        \item Ứng dụng trên tập dữ liệu FER-2013  
        \item Tích hợp với mô hình MobileNetV3  
    \end{itemize}

    \item \textbf{Chương 4 – Thực nghiệm và đánh giá}  
    \begin{itemize}
        \item Thiết lập huấn luyện và tham số mô hình  
        \item Đánh giá theo các chỉ số: Accuracy, F1-score, v.v.  
        \item So sánh mô hình có và không có tăng cường thích ứng  
    \end{itemize}

    \item \textbf{Chương 5 – Kết luận và hướng phát triển}  
    \begin{itemize}
        \item Tổng kết nội dung đã thực hiện  
        \item Đánh giá ưu, nhược điểm  
        \item Đề xuất hướng phát triển trong tương lai  
    \end{itemize}
\end{itemize}