\section*{MỤC TIÊU VÀ PHẠM VI NGHIÊN CỨU}

\begin{tabular}{|p{4cm}|p{11cm}|}
    \hline
    \textbf{Câu hỏi gợi ý} & Mục tiêu nghiên cứu chính của đề tài là gì? Phạm vi nghiên cứu là gì? \\
    \hline
    \textbf{Hướng dẫn} & Đặt bài toán giải quyết và trình bày mục tiêu nghiên cứu chính của đề tài. Nêu phạm vi nghiên cứu của đề tài, bao gồm việc giới hạn phạm vi nghiên cứu và triển khai, các giả định ban đầu đối với nghiên cứu. \\
    \hline
\end{tabular}

\medskip

\textbf{Mục tiêu nghiên cứu:} Đề tài hướng đến giải quyết bài toán nhận diện biểu cảm khuôn mặt trong điều kiện ánh sáng yếu, một thách thức thực tiễn trong các ứng dụng như giám sát an ninh và giao tiếp người-máy. Các mục tiêu chính bao gồm:  
\begin{enumerate}
    \item Xây dựng mô hình CNN nhẹ (MobileNetV3) để nhận diện biểu cảm khuôn mặt trong điều kiện ánh sáng yếu.
    \item Đề xuất và triển khai kỹ thuật tăng cường dữ liệu thích ứng dựa trên mức độ sáng của từng ảnh nhằm nâng cao hiệu suất mô hình.
    \item Đánh giá hiệu quả của phương pháp đề xuất so với các phương pháp cơ bản không sử dụng tăng cường dữ liệu thích ứng.
\end{enumerate}

\textbf{Phạm vi nghiên cứu:} Đề tài được triển khai và thử nghiệm trên tập dữ liệu FER-2013, với trọng tâm là xử lý điều kiện ánh sáng yếu. Nghiên cứu giới hạn trong việc sử dụng CNN nhẹ (MobileNetV3) và kỹ thuật tăng cường dữ liệu thích ứng, không mở rộng sang các mô hình phức tạp như GAN hay Retinex. Các giả định ban đầu bao gồm:  
\begin{itemize}
    \item Tập dữ liệu FER-2013 sau khi được biến đổi để mô phỏng ánh sáng yếu vẫn đủ đại diện cho bài toán thực tế.
    \item Kỹ thuật tăng cường dữ liệu thích ứng có thể cải thiện hiệu suất của mô hình CNN nhẹ so với phương pháp cố định.
\end{itemize}
