\section{Dự kiến kế hoạch nghiên cứu} % Section 8

Kế hoạch thực hiện đề tài được chia thành các giai đoạn rõ ràng, từ nghiên cứu lý thuyết, thu thập dữ liệu, đến thực nghiệm và viết báo cáo. Nhóm dự kiến hoàn thành trong thời gian 8 tuần như sau:

\begin{center}
\renewcommand{\arraystretch}{1.5}
\begin{tabular}{|c|p{8cm}|c|}
    \hline
    \textbf{STT} & \centering \textbf{Nội dung công việc} & \textbf{Thời gian} \\ \hline
    1 & Tìm hiểu tổng quan về nhận diện biểu cảm khuôn mặt, điều kiện ánh sáng yếu và các mô hình CNN nhẹ & Tuần 1 \\ \hline
    2 & Khảo sát kỹ thuật tăng cường dữ liệu và đề xuất phương pháp tăng cường thích ứng theo ánh sáng ảnh & Tuần 2 \\ \hline
    3 & Tiền xử lý dữ liệu FER-2013 và xây dựng tập dữ liệu mô phỏng điều kiện ánh sáng yếu & Tuần 3 \\ \hline
    4 & Huấn luyện mô hình MobileNetV3 cơ bản và tinh chỉnh tham số & Tuần 4 \\ \hline
    5 & Tích hợp kỹ thuật tăng cường thích ứng vào pipeline huấn luyện & Tuần 5 \\ \hline
    6 & Đánh giá mô hình: so sánh với mô hình không tăng cường; phân tích kết quả qua các chỉ số & Tuần 6 \\ \hline
    7 & Viết báo cáo, chuẩn bị biểu đồ, bảng số liệu, hình ảnh minh họa & Tuần 7 \\ \hline
    8 & Rà soát, chỉnh sửa báo cáo, hoàn thiện đề cương và chuẩn bị bảo vệ & Tuần 8 \\ \hline
\end{tabular}
\end{center}

Kế hoạch có thể được điều chỉnh linh hoạt tùy theo tiến độ thực tế, tuy nhiên nhóm sẽ đảm bảo hoàn thành đúng hạn và chất lượng tốt nhất.