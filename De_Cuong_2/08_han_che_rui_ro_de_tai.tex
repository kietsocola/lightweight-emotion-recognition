\section{Hạn chế và rủi ro của đề tài} % Section 8

Mặc dù đề tài đã được định hướng rõ ràng, phù hợp với năng lực thực hiện và có tính khả thi cao, nhóm nghiên cứu vẫn nhận thấy một số hạn chế và rủi ro có thể ảnh hưởng đến kết quả cuối cùng như sau:

\begin{itemize}
    \item \textbf{Dữ liệu ánh sáng yếu chỉ mang tính mô phỏng:} Do hạn chế về nguồn dữ liệu thực tế, nhóm sử dụng ảnh FER-2013 và áp dụng kỹ thuật giảm sáng để tạo ra tập dữ liệu ánh sáng yếu. Điều này có thể không phản ánh đầy đủ các tình huống ánh sáng yếu thực tế ngoài môi trường.

    \item \textbf{Khả năng tổng quát hóa của mô hình còn hạn chế:} Mô hình được huấn luyện trên dữ liệu tăng cường có thể chưa phản ánh được đầy đủ các biểu cảm phức tạp, đặc biệt là biểu cảm vi mô hoặc biểu cảm có sự che khuất khuôn mặt.

    \item \textbf{Giới hạn về tài nguyên tính toán:} Việc huấn luyện và thử nghiệm mô hình chủ yếu được thực hiện trên CPU hoặc môi trường giả lập, chưa kiểm nghiệm trên các thiết bị thực tế như camera nhúng hoặc hệ thống IoT.

    \item \textbf{Rủi ro về thời gian triển khai:} Việc xử lý tập dữ liệu lớn, tinh chỉnh mô hình và đánh giá kết quả có thể kéo dài hơn dự kiến nếu gặp lỗi kỹ thuật hoặc thiếu tài nguyên máy tính.
\end{itemize}

Nhóm sẽ cố gắng khắc phục và giảm thiểu các hạn chế trên bằng cách tối ưu quy trình huấn luyện, kiểm thử cẩn thận, và chuẩn bị phương án dự phòng trong quá trình triển khai.