


Trong bài báo gốc ``Ames, Iowa: Alternative to the Boston Housing Data as an End of Semester Regression Project'' (\textit{Journal of Statistics Education}, 2011), Dean De Cock đã giới thiệu tập dữ liệu Ames Housing và thử nghiệm các mô hình hồi quy để dự đoán giá bán nhà (\texttt{SalePrice}). Các mô hình chính được De Cock áp dụng bao gồm:

\textbf{Hồi quy Tuyến tính Đa biến (Multiple Linear Regression):} \\
De Cock sử dụng mô hình hồi quy tuyến tính đa biến làm phương pháp chính để phân tích mối quan hệ giữa các biến độc lập (như diện tích nhà, số phòng, chất lượng tổng thể) và biến phụ thuộc (giá bán nhà). Kết quả cho thấy mô hình này có thể giải thích một phần đáng kể sự biến thiên của giá nhà, với hệ số xác định $R^2$ đạt khoảng $0.7-0.8$ khi dữ liệu được tiền xử lý tốt. Tuy nhiên, ông lưu ý rằng mô hình này không đủ mạnh để nắm bắt các mối quan hệ phi tuyến tính.

\textbf{Phân tích Hồi quy Cơ bản:} \\
Bài báo tập trung vào việc sử dụng tập dữ liệu như một công cụ giảng dạy cho sinh viên, nên De Cock chủ yếu áp dụng các kỹ thuật hồi quy cơ bản (\textit{basic regression techniques}) để minh họa cách xây dựng mô hình dự đoán. Ông không đi sâu vào các mô hình phức tạp như Random Forest hay Gradient Boosting, mà ưu tiên các phương pháp đơn giản, dễ hiểu cho mục đích giáo dục.

Link bài báo: \url{https://jse.amstat.org/v19n3/decock.pdf}
