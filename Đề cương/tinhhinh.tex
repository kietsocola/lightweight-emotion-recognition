\section*{TÌNH HÌNH NGHIÊN CỨU TRONG VÀ NGOÀI NƯỚC}
\subsection*{Câu hỏi gợi ý}
Lĩnh vực và nghiên cứu tiên quan đã và đang phát triển như thế nào? Các vấn đề, và bài toán đặt ra cần giải quyết là gì?
\subsection*{Hướng dẫn}
Phát triển khá lâu dài. Bắt nguồn từ các hình thức đào tạo như học tập có sự hỗ trợ của máy tính (Computer-assisted learning), đào tạo dựa trên máy tính (Computer-Based Training) khá phổ biến trong các thập kỷ 70, 80 của thế kỷ 20 [14], e-Learning hiện nay đã chịu sự phạm phù hợp với từng ngữ cảnh, áp dụng mô hình dạy học kết hợp để tăng hiệu quả đào tạo.

Tuy góp phần làm thay đổi hành vi học tập của người học và mở ra khả năng tiếp cận tri thức vô cùng to lớn cho nhiều đối tượng người học khác nhau, nhưng e-Learning cũng đã phát sinh khá nhiều hạn chế. Một trong những hạn chế đó là lỗi thiết kế theo kiểu “one size fits all”, đánh đồng các người học với nhau mà không biết rằng, mỗi người học sẽ có nhu cầu học tập khác nhau, trình độ nhận thức và sở thích rất khác nhau. Với kiểu thiết kế như vậy, người học sẽ không cảm thấy hứng thú và gắn kết với hệ thống, điều này làm ảnh hưởng đến kết quả học tập và phát sinh tư tưởng học đối phó.

Gần đây, một thiết kế e-Learning mới ra đời về cơ bản có thể xóa bỏ tình trạng này. Đó là các hệ thống học tập thích nghi (Adaptive e-Learning System). Các hệ này vốn bắt nguồn từ lĩnh vực thương mại điện tử để đưa ra các lời tư vấn dành cho khách hàng. Với ứng dụng trong giáo dục, hệ thích nghi tạo ra các tư vấn cho người học về nội dung kiến thức cần học trong một khóa học cụ thể, hoặc tư vấn cho các người học khác nhau phương pháp học phù hợp với trình độ và khả năng tiếp thu của từng người. Trong các hệ thống thích nghi này, mỗi người học sở hữu một thành phần mô tả đặc trưng người học (profile). Đặc trưng người học chính là cơ sở để hệ thống cung cấp những thông tin, dịch vụ, tài nguyên, phù hợp với từng người học. Điều này đem đến sự tiện nghi, thoải mái cho người học trong quá trình học tập trên hệ thống. Người học có cảm giác là hệ thống rất thông minh, hiểu được mình và đáp ứng đúng nhu cầu năng của mình [32].

Tại Việt Nam, e-Learning đã được quan tâm nghiên cứu từ những năm đầu của thế kỷ 21. Tuy nhiên, các nghiên cứu e-Learning như Nguyễn Việt Anh [34], Lê Đức Long [36], trong các nghiên cứu luận án tiến sĩ của mình, đã có đề cập đến hệ thống học tập thích nghi, nhưng về mặt ứng dụng thì hiện vẫn chưa có một hệ thống học tập trực tuyến nào - theo kiểu thiết kế thích nghi - được xây dựng và khai thác. Do vậy, trong thời gian sắp tới, e-Learning trong nước vẫn còn phải đối mặt với nhiều khó khăn và thách thức.

\textbf{Ghi chú}