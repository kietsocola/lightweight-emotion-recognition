\documentclass{article}
\usepackage[T1]{fontenc}  
\usepackage[utf8]{inputenc}  
\usepackage[vietnamese]{babel}  
\usepackage{graphicx}  
\usepackage{hyperref}
\usepackage[utf8]{vietnam}
\usepackage{amsmath}
\usepackage{enumitem}
\usepackage{tikz} % Thêm gói tikz để vẽ khung
\usepackage{tabularx}

\begin{document}

% Trang bìa với khung và nội dung
\begin{titlepage}
    \begin{tikzpicture}[remember picture, overlay]
        % Vẽ khung trang trí gần mép trang
        \draw[thick, rounded corners, draw=blue!50!black] 
            ([shift={(1cm,-1cm)}]current page.north west) 
            rectangle 
            ([shift={(-1cm,1cm)}]current page.south east);
        % Vẽ khung bên trong (tùy chọn)
        \draw[dashed, draw=red!70!black] 
            ([shift={(1.5cm,-1.5cm)}]current page.north west) 
            rectangle 
            ([shift={(-1.5cm,1.5cm)}]current page.south east);
    \end{tikzpicture}
    \centering
{\LARGE \textbf{TRƯỜNG ĐẠI HỌC SÀI GÒN}} \\[0.2cm]
{\Large \textbf{KHOA CÔNG NGHỆ THÔNG TIN}} \\[0.5cm]

\vspace{0.5cm} 
\includegraphics[width=4cm]{./img/logo.png} % Chèn logo 
\vspace{0.5cm} % Giảm khoảng cách

{\huge \textbf{HỌC PHẦN: NCKH TRONG CNTT}} \\[0.5cm]
{\Large KHẢO SÁT BÀI BÁO CHO DỮ LIỆU AMES HOUSING} \\[0.5cm]

\vspace{0.5cm} % Giảm khoảng cách

\textbf{Nhóm sinh viên thực hiện:} \\[0.5cm]

\begin{tabular}{|l|l|}
    \hline
    \textbf{Họ và tên} & \textbf{MSSV} \\ \hline
    Văn Tuấn Kiệt & 3122410202 \\ \hline
    Mai Phúc Lâm & 3122410207 \\ \hline
    Nguyễn Đức Duy Lâm & 3122410208 \\ \hline
    Nguyễn Hữu Lộc & 3122410213 \\ \hline
\end{tabular}

\vspace{0.5cm}

\textbf{Giáo viên hướng dẫn:} Đỗ Như Tài \\[0.5cm]

\vfill
{\Large TP.HCM, 2025} % Gọi nội dung từ header/title.tex
\end{titlepage}

\newpage

\section*{Phân công công việc} % Tiêu đề của trang mới

\vspace{0.5cm} % Giảm khoảng cách

\begin{table}[h]
    \centering
    \renewcommand{\arraystretch}{1.3}
    \begin{tabularx}{\textwidth}{|c|c|X|X|c|}
        \hline
        \textbf{STT} & \textbf{MSSV} & \textbf{Họ và Tên} & \textbf{Phân Công} & \textbf{Thái Độ} \\ 
        \hline
        1 & 3122410202 & Văn Tuấn Kiệt & Các bài toán liên quan & Rất tốt \\ 
        \hline
        2 & 3122410207 & Mai Phúc Lâm & Nguồn gốc và vị trí của tập dữ liệu & Tích cực \\ 
        \hline
        3 & 3122410208 & Nguyễn Đức Duy Lâm & Kết quả đạt được, độ đo, khảo sát, kết luận & Nhiệt tình \\ 
        \hline
        4 & 3122410213 & Nguyễn Hữu Lộc & Tóm tắt, giới thiệu, nguồn gốc dữ liệu & Trách nhiệm \\ 
        \hline
    \end{tabularx}
    \caption{Phân công công việc nhóm}
    \label{tab:phancong}
\end{table}

\vspace{0.5cm}

\newpage
% Mục lục
\tableofcontents
\newpage

% Phần 1: Title
\begin{center}
    \textbf{\Large Phân tích luận văn tốt nghiệp} \\
    \vspace{0.5cm}
    \normalsize Tác giả: Văn Tuấn Kiệt$^1$, Mai Phúc Lâm$^1$, Nguyễn Đức Duy Lâm$^1$, Nguyễn Hữu Lộc$^1$ \\
    \vspace{0.2cm}
    $^1$Trường Đại học Sài Gòn,
    \vspace{0.2cm}
    Ngày: 17 tháng 4 năm 2025 \\
    \vspace{0.2cm}
\end{center}

\section{Giới thiệu}
\section*{GIỚI THIỆU}
\subsection*{Câu hỏi gợi ý}
\textbf{Hướng dẫn}

Ngày nay, công nghệ đã trở thành một yếu tố quan trọng làm thay đổi cách sống, cách nghĩ, cách làm việc và cách tiếp cận văn hóa của con người. Thật vậy, với sự phát triển như vũ bão của công nghệ ở thế kỷ 21, đặc biệt là công nghệ thông tin và truyền thông, viết tắt là ICT, con người đã tạo ra được những công cụ mới phục vụ tốt hơn cho cuộc sống của mình. ICT có mặt ở khắp mọi nơi, mọi lĩnh vực, từ thương mại, y tế, văn hóa, chính trị,... và giáo dục cũng không phải là ngoại lệ. Phải khẳng định rằng, để phát triển đất nước, tất yếu phải phát triển giáo dục, và giáo dục phải đi trước một bước hơn tất cả. Để làm được điều đó, sự hỗ trợ từ ICT dành cho giáo dục là hết sức cần thiết. Từ lâu, các nhà nghiên cứu giáo dục đã nghiên cứu cách thức áp dụng ICT để nâng cao chất lượng giáo dục, đưa công nghệ thâm nhập sâu hơn vào giáo dục, tạo ra các công cụ giáo dục mới, có chất lượng tốt hơn hẳn. Các nghiên cứu đã chỉ ra rằng tầng e-Learning mang lại nhiều lợi ích cho hoạt động giảng dạy bởi việc trợ giúp giảng viên và học viên đạt được những kỹ năng cần thiết cho công việc ở thế kỷ 21 [13][29][12]. Tuy nhiên, việc ứng dụng e-Learning trong các hệ thống học tập trực tuyến vẫn còn nhiều vấn đề phức tạp cần phải nghiên cứu đối với đa số những nhà giáo dục, những chuyên gia trong lĩnh vực này [1]. Tại Việt Nam, e-Learning đã được nghiên cứu và tiếp cận bởi khá nhiều trường đại học. Các trường này đã cố gắng xây dựng cho riêng mình những hệ thống học tập trực tuyến để hỗ trợ cho hoạt động giảng dạy hiện tại hoặc phục vụ đào tạo từ xa. Bên cạnh các thế, chủ yếu do vấn đề tương tác giữa người học với giáo viên và người học với hệ thống. Do vậy, bài toán đặt ra là có thể xây dựng một hệ e-Learning tiếp cận theo hướng thích nghi phù hợp với ngữ cảnh dạy học tại Việt Nam mà cụ thể là áp dụng tại Trường Đại học Sư phạm TPHCM được hay không?

\textbf{Ghi chú}
\section{Mục tiêu nghiên cứu}
Mục tiêu của luận văn là phát triển và đánh giá một phương pháp nhận diện cảm xúc trên khuôn mặt người sử dụng mạng học sâu tích hợp cơ chế chú ý (Residual Masking Network) để phân loại cảm xúc từ hình ảnh đầu vào trong môi trường phức tạp, đồng thời sử dụng phương pháp học kết hợp nhiều mô hình hiện đại nhằm nâng cao độ chính xác.
\section{Cơ sở lý thuyết}
\begin{itemize}
    \item Luận văn dựa trên các nghiên cứu về nhận diện cảm xúc khuôn mặt, một lĩnh vực đã được phát triển nhiều năm với các lợi ích trong nhiều ứng dụng thực tiễn.
    \item Sử dụng mạng nơ-ron tích chập (CNN) và cơ chế chú ý để tập trung vào các đặc trưng quan trọng trên khuôn mặt liên quan đến cảm xúc (theo Hệ thống mã hóa hành động khuôn mặt - FACS).
    \item Các mô hình hiện đại như VGG19, ResNet, DenseNet, GoogLeNet, Inception v3, 
    \item Dữ liệu được lấy từ các tập dữ liệu chuẩn như FER2013 và VEMO, chứa các hình ảnh khuôn mặt với nhãn cảm xúc cơ bản (giận dữ, ghê tởm, sợ hãi, hạnh phúc, buồn bã, ngạc nhiên, trung lập).
\end{itemize}
\section{Phương pháp thu thập và sử lí thông tin}
\begin{itemize}
    \item Sử dụng tập dữ liệu FER2013 và VEMO. Các hình ảnh được xử lý để phát hiện khuôn mặt bằng phương pháp của OpenCV 3.4 nếu chưa được xử lý trước.
    \item Áp dụng hai phép biến đổi chính: lật ảnh theo trục dọc (Flip Left Right) và xoay ảnh từ -30 đến 30 độ, sử dụng thư viện imgaug.
    \item Điều chỉnh kích thước ảnh về 224x224 pixel, chuyển đổi sang tensor, và nhân bản thành ba kênh màu
    \item Sử dụng lớp Dataset trong PyTorch để tải và quản lý dữ liệu, với hàm getitem để truy xuất ảnh và nhãn.
    \item Sử dụng mạng Residual Masking Network (ResMaskingNet) với cơ chế chú ý để phân loại cảm xúc.
    \item Sử dụng GradCAM và gộp theo chiều kênh (Average Pooling) để trực quan hóa các vùng chú ý của mô hình trên khuôn mặt.
\end{itemize}
\section{Kết quả đạt được}
\subsection{Trên tập FER2013}
\begin{itemize}
    \item Mạng ResMaskingNet đạt độ chính xác 74.14\%, vượt qua nhiều mô hình hiện đại như VGG19 (70.8\%), ResNet18 (72.9\%), DenseNet121 (73.16\%), và CBAM\_ResNet50 (73.39\%).
    \item Khi kết hợp ResMaskingNet với 6 mô hình CNN khác, độ chính xác đạt 76.82\%, là kết quả tốt nhất so với các phương pháp được báo cáo khoa học (ví dụ: Ensemble 8 CNNs đạt 75.2\%).
\end{itemize}
\subsection{Trên tập VEMO}
\begin{itemize}
    \item ResMaskingNet đạt độ chính xác 65.94\%, vượt qua ResNet18 (63.94\%), ResNet34 (64.84\%), và ResAttNet56 (60.82\%).
\end{itemize}
\subsection{Trực quan hoá GradCAM}
\begin{itemize}
    \item Chúng đã tập trung vào vùng mặt người, có chú ý vào các bộ phận có ảnh hưởng trực tiếp đến cảm xúc như được miêu tả trong FACS
    \item Mô hình được thử nghiệm trên các hình ảnh thực tế (ví dụ: ảnh từ phim "Mắt Biếc") và cho kết quả dự đoán chính xác cảm xúc như hạnh phúc.
\end{itemize}


\section{Hạn chế công trình}
\begin{itemize}
    \item Chỉ sử dụng 2 phương pháp tiền xử lí dữ liệu
    \item Khi trực quan hoá bằng gộp kênh thì chúng không thể hiện rõ ràng sự khác nhau giữa các đặc trưng cho ta thấy
    \item Qua ma trận nhầm lẫn , việc dự đoán cảm xúc trên các tập dữ liệu dễ bị nhầm lần fear  , disgust
\end{itemize}

\end{document}
