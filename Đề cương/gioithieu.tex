\section*{GIỚI THIỆU}

\begin{tabular}{|p{4cm}|p{11cm}|}
\hline
\textbf{Câu hỏi gợi ý} & Lĩnh vực và nghiên cứu liên quan đã và đang phát triển như thế nào? Các vấn đề, và bài toán đặt ra cần giải quyết là gì? \\
\hline
\textbf{Hướng dẫn} & Giới thiệu tổng quan về đề tài – những vấn đề và lĩnh vực liên quan đến đề tài. \\
\hline
\end{tabular}

\medskip

Ngày nay, công nghệ trí tuệ nhân tạo (AI) đã trở thành một yếu tố quan trọng trong việc thay đổi cách con người tiếp cận và giải quyết các vấn đề thực tiễn. Đặc biệt, nhận diện biểu cảm khuôn mặt (Facial Expression Recognition - FER) là một lĩnh vực có tiềm năng ứng dụng lớn trong giao tiếp người-máy, giám sát an ninh, và phân tích cảm xúc. Tuy nhiên, hiệu suất của các mô hình học sâu như CNN thường giảm đáng kể trong điều kiện ánh sáng yếu - một thách thức phổ biến trong các ứng dụng thực tế như camera giám sát ban đêm hoặc thiết bị nhúng.  

Các nghiên cứu gần đây (2020-2025) đã đề xuất nhiều phương pháp để cải thiện FER trong điều kiện ánh sáng yếu, chẳng hạn như sử dụng GAN hoặc Retinex, nhưng các giải pháp này thường phức tạp và không khả thi với nguồn lực hạn chế. Do đó, bài toán đặt ra là liệu có thể xây dựng một hệ thống nhận diện biểu cảm đơn giản, hiệu quả, sử dụng CNN nhẹ (MobileNetV3) kết hợp với kỹ thuật tăng cường dữ liệu thích ứng hay không?