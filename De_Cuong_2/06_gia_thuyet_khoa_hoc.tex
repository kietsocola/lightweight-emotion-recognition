\section{Giả thuyết khoa học} % Section 6

Dựa trên các nghiên cứu trước và định hướng của đề tài, nhóm đưa ra các giả thuyết khoa học như sau:

\begin{itemize}
    \item \textbf{Giả thuyết 1:} Mô hình học sâu nhẹ (MobileNetV3) có thể đạt hiệu suất nhận diện biểu cảm khuôn mặt tốt trong điều kiện ánh sáng yếu nếu được huấn luyện với tập dữ liệu được tăng cường thích ứng theo mức độ sáng.
    
    \item \textbf{Giả thuyết 2:} Kỹ thuật tăng cường dữ liệu ảnh thích ứng theo đặc trưng ánh sáng đầu vào giúp cải thiện độ chính xác và khả năng khái quát hóa của mô hình so với các phương pháp tăng cường cố định truyền thống.
    
    \item \textbf{Giả thuyết 3:} Hệ thống kết hợp giữa MobileNetV3 và tăng cường dữ liệu thích ứng có thể duy trì độ chính xác trên 70\% và tốc độ suy luận dưới 0.1 giây/ảnh trên CPU, phù hợp với yêu cầu triển khai thực tế trên thiết bị giới hạn tài nguyên.
\end{itemize}

Các giả thuyết trên sẽ được kiểm chứng thông qua thực nghiệm trên tập dữ liệu FER-2013 đã xử lý, bằng các chỉ số đánh giá như Accuracy, Precision, Recall, F1-score, và thời gian xử lý.