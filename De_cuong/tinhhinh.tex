\section*{TÌNH HÌNH NGHIÊN CỨU TRONG VÀ NGOÀI NƯỚC}

\begin{tabular}{|p{4cm}|p{11cm}|}
    \hline
    \textbf{Câu hỏi gợi ý} & Lĩnh vực và nghiên cứu liên quan đã và đang phát triển như thế nào? Các vấn đề, và bài toán đặt ra cần giải quyết là gì? \\
    \hline
    \textbf{Hướng dẫn} & Tìm hiểu các nghiên cứu đã công bố gần nhất (5 năm trở lại) về lĩnh vực liên quan trong và ngoài nước. Phân tích các kết quả đóng góp, nhận xét các hạn chế còn tồn tại. \\
    \hline
\end{tabular}
    
\medskip

\subsection*{Nghiên cứu ngoài nước}
Trong những năm gần đây, nhận diện cảm xúc khuôn mặt (FER) nổi bật trong AI và thị giác máy tính nhờ học sâu. Các mô hình CNN như VGGNet (73,28\% - Khaireddin et al.), ResNet, và ensemble (75,8\% - Khanzada et al.) vượt kỷ lục trên FER2013. Ứng dụng mở rộng trong y tế, giáo dục, an ninh, và HCI, ví dụ web app thời gian thực (69,8\%, 40ms). Nghiên cứu xử lý dữ liệu phức tạp qua dữ liệu phụ (CK+, JAFFE), tăng cường dữ liệu, và đa phương thức, đạt 99,26\% trên CK+ (AA-DCN, 2024). Thách thức gồm thiếu dữ liệu đa dạng và vấn đề đạo đức, hướng tới tích hợp mốc khuôn mặt, CNN chú ý, và dữ liệu lớn như AffectNet.

\subsection*{Nghiên cứu trong nước (Việt Nam)}
Tại Việt Nam, nghiên cứu về nhận diện cảm xúc khuôn mặt tuy còn ở giai đoạn phát triển nhưng đã ghi nhận những kết quả đáng chú ý, tập trung vào ứng dụng thực tiễn: Công nghệ FER được áp dụng trong giám sát giao thông (phát hiện tài xế mệt mỏi), giáo dục (phân tích phản ứng học sinh qua video) và dịch vụ khách hàng (đánh giá mức độ hài lòng).Cũng đạt được những kết quả ấn tượng 
\section*{Phân tích kết quả đóng góp}

\subsection*{Đóng góp chung}
\begin{itemize}
    \item Phát triển công nghệ học sâu, nhận diện cảm xúc chính xác hơn.
    \item Ứng dụng đa dạng: y tế, giáo dục, an ninh, tiếp thị.
    \item Cung cấp tập dữ liệu chuẩn và hiểu biết về hành vi con người.
\end{itemize}

\subsection*{Hạn chế chung}
\begin{itemize}
    \item  Hiệu suất giảm trong điều kiện ánh sáng yếu do mất mát chi tiết và độ sáng thấp.
    \item  Các yếu tố gây nhiễu trong môi trường không kiểm soát (che khuất, tư thế, v.v.).
    \item  Hiệu suất tính toán hạn chế của các mô hình phức tạp, không phù hợp với thiết bị tài nguyên thấp.
    \item  Thiếu dữ liệu ánh sáng yếu thực tế, dẫn đến việc phải sử dụng dữ liệu tổng hợp.
    \item  Sự mất cân bằng dữ liệu và nhầm lẫn giữa các cảm xúc tương đồng.
  
\end{itemize}

\section*{Kết luận}
Nghiên cứu về nhận diện cảm xúc khuôn mặt trong và ngoài nước đã đạt được nhiều thành tựu quan trọng, từ cải tiến thuật toán đến ứng dụng thực tiễn. Tuy nhiên, để phát triển bền vững, cần giải quyết các vấn đề về quyền riêng tư, thiên vị thuật toán và dữ liệu đa dạng, đặc biệt tại Việt Nam, nơi tiềm năng nghiên cứu còn lớn nhưng nguồn lực còn hạn chế.