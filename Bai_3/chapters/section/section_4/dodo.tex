
Khi đánh giá hiệu suất của các mô hình dự đoán giá nhà trong tập dữ liệu Ames Housing, các độ đo sau thường được sử dụng:

\textbf{$R^2$ (Hệ số Xác định - Coefficient of Determination):} \\
Đo lường mức độ mô hình giải thích được sự biến thiên của giá bán nhà (\texttt{SalePrice}). Trong bài báo của De Cock (2011), $R^2$ được sử dụng để đánh giá mô hình hồi quy tuyến tính đa biến, đạt khoảng $0.7-0.8$ với dữ liệu tiền xử lý tốt. Giá trị này phản ánh khả năng giải thích của mô hình cơ bản.

\textbf{RMSE (Root Mean Squared Error - Căn bậc hai Sai số Bình phương Trung bình):} \\
Đo sai số trung bình giữa giá trị dự đoán và thực tế. De Cock không báo cáo RMSE cụ thể trong bài báo, nhưng đây là độ đo phổ biến trong các nghiên cứu sau này (ví dụ: trên Kaggle), với giá trị thường từ $0.11-0.15$ khi giá nhà được biến đổi log.

\textbf{MAE (Mean Absolute Error - Sai số Tuyệt đối Trung bình):} \\
Đo sai số tuyệt đối trung bình, ít nhạy với giá trị ngoại lai hơn RMSE. De Cock không đề cập trực tiếp MAE, nhưng nó thường được dùng bổ sung trong các phân tích thực tế, với kết quả khoảng $10,000-20,000$ USD trên dữ liệu gốc.

Link bài báo: \url{https://jse.amstat.org/v19n3/decock.pdf}
