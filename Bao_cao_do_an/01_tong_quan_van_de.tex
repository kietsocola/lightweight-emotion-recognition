\section{Tổng quan vấn đề} % Section 1
\subsection{Lý do chọn đề tài}
Nhận diện biểu cảm khuôn mặt (Facial Expression Recognition - FER) đóng vai trò quan trọng trong các ứng dụng thực tiễn như giao tiếp người-máy, giám sát an ninh, và phân tích hành vi. Tuy nhiên, trong các điều kiện ánh sáng yếu, chẳng hạn như môi trường ban đêm hoặc khu vực thiếu sáng, hiệu quả của các hệ thống FER giảm đáng kể do chất lượng hình ảnh thấp. Các nghiên cứu gần đây (2020--2025) chủ yếu tập trung vào điều kiện ánh sáng lý tưởng, trong khi các giải pháp cho ánh sáng yếu thường phức tạp, đòi hỏi tài nguyên tính toán lớn hoặc không tối ưu cho các thiết bị nhúng. 

Việc phát triển một phương pháp nhận diện biểu cảm hiệu quả trong điều kiện ánh sáng yếu, sử dụng mô hình CNN nhẹ (như MobileNetV3) và kỹ thuật tăng cường dữ liệu thích ứng, không chỉ đáp ứng nhu cầu thực tiễn mà còn mang lại giá trị khoa học thông qua việc cải tiến các kỹ thuật hiện có. Đề tài này được chọn vì tính khả thi trong thời gian nghiên cứu (6 tuần), tính mới trong việc kết hợp các phương pháp đơn giản nhưng hiệu quả, và tiềm năng ứng dụng trong các hệ thống thực tế như camera giám sát hoặc thiết bị IoT.

% Subsection: Vấn đề nghiên cứu
\subsection{Vấn đề nghiên cứu}
Trong điều kiện ánh sáng yếu, các mô hình nhận diện biểu cảm khuôn mặt truyền thống thường gặp khó khăn do độ tương phản thấp, nhiễu ảnh, và mất chi tiết khuôn mặt. Các phương pháp hiện tại như sử dụng GAN (Generative Adversarial Networks) hoặc Retinex-based preprocessing tuy hiệu quả nhưng phức tạp, yêu cầu thời gian huấn luyện lâu và tài nguyên tính toán lớn, không phù hợp với các ứng dụng thời gian thực hoặc thiết bị có tài nguyên hạn chế. Ngoài ra, các kỹ thuật tăng cường dữ liệu cố định (fixed augmentation) không tối ưu vì không thích nghi với mức độ ánh sáng yếu khác nhau của từng ảnh.

Vấn đề nghiên cứu được đặt ra là: Làm thế nào để phát triển một hệ thống nhận diện biểu cảm khuôn mặt trong điều kiện ánh sáng yếu, sử dụng mô hình CNN nhẹ và kỹ thuật tăng cường dữ liệu thích ứng, nhằm đạt được độ chính xác cao, tốc độ xử lý nhanh, và khả năng triển khai trên các thiết bị nhúng?

% Subsection: Mục tiêu nghiên cứu
\subsection{Mục tiêu nghiên cứu}
Mục tiêu tổng quát của nghiên cứu là xây dựng một hệ thống nhận diện biểu cảm khuôn mặt hiệu quả trong điều kiện ánh sáng yếu, sử dụng mạng nơ-ron tích chập nhẹ (MobileNetV3) kết hợp với kỹ thuật tăng cường dữ liệu thích ứng. Các mục tiêu cụ thể bao gồm:
\begin{enumerate}
    \item Phát triển một pipeline tăng cường dữ liệu thích ứng, tự động điều chỉnh các kỹ thuật tăng cường dựa trên mức độ ánh sáng yếu của từng ảnh.
    \item Huấn luyện và tinh chỉnh mô hình MobileNetV3 để nhận diện biểu cảm khuôn mặt trong điều kiện ánh sáng yếu với độ chính xác cao.
    \item Đánh giá và so sánh hiệu quả của phương pháp đề xuất với các kỹ thuật tăng cường dữ liệu cố định và các mô hình CNN khác (nếu khả thi).
\end{enumerate}

% Subsection: Câu hỏi nghiên cứu
\subsection{Câu hỏi nghiên cứu}
Nghiên cứu tập trung trả lời các câu hỏi sau:
\begin{enumerate}
    \item Làm thế nào để thiết kế một pipeline tăng cường dữ liệu thích ứng, hiệu quả trong việc cải thiện chất lượng ảnh ánh sáng yếu cho nhận diện biểu cảm khuôn mặt?
    \item Mô hình MobileNetV3 có thể đạt được độ chính xác tương đương hoặc vượt trội so với các kỹ thuật tăng cường dữ liệu cố định trong điều kiện ánh sáng yếu không?
    \item Các kỹ thuật tăng cường dữ liệu thích ứng ảnh hưởng như thế nào đến hiệu suất của mô hình CNN nhẹ trong nhận diện biểu cảm khuôn mặt?
\end{enumerate}

% Subsection: Phạm vi nghiên cứu
\subsection{Phạm vi nghiên cứu}
\begin{itemize}
    \item \textbf{Đối tượng nghiên cứu}: Các biểu cảm khuôn mặt (ví dụ: vui, buồn, tức giận, ngạc nhiên) trong điều kiện ánh sáng yếu, được mô phỏng hoặc thu thập từ bộ dữ liệu công khai FER-2013.
    \item \textbf{Phạm vi không gian}: Nghiên cứu tập trung vào xử lý hình ảnh tĩnh (static images), không bao gồm dữ liệu video hoặc dữ liệu đa phổ.
    \item \textbf{Phạm vi thời gian}: Nghiên cứu được thực hiện trong 8 tuần, từ tháng 4 đến tháng 5 năm 2025, với các thí nghiệm dựa trên dữ liệu công khai và mô hình pre-trained.
    \item \textbf{Phạm vi kỹ thuật}: Sử dụng mô hình CNN nhẹ (MobileNetV3) và các kỹ thuật tăng cường dữ liệu như gamma correction, histogram equalization, được triển khai bằng Python với các thư viện TensorFlow/Keras và OpenCV.
\end{itemize}