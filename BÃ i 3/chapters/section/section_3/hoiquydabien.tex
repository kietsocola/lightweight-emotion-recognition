Tập dữ liệu Ames Housing thường được sử dụng để giải quyết bài toán hồi quy đa biến, trong đó mục tiêu là dự đoán giá bán nhà (SalePrice) dựa trên nhiều biến độc lập như diện tích (GrLivArea), số lượng phòng ngủ (BedroomAbvGr), hay chất lượng tổng thể của ngôi nhà (OverallQual). Theo De Cock (2011, trang 10), tập dữ liệu này cung cấp tới 79 biến giải thích, tạo điều kiện để áp dụng các mô hình hồi quy tuyến tính đa biến hoặc các mô hình phức tạp hơn như hồi quy Ridge và Lasso. Một ví dụ cụ thể là nghiên cứu của Pace và Barry (1997) về phân tích không gian trong định giá bất động sản, trong đó các biến liên quan đến vị trí như Neighborhood trong Ames Housing có thể được sử dụng để cải thiện độ chính xác của dự đoán.\\
Link bài báo: \href{https://www.sciencedirect.com/science/article/abs/pii/S016771529600140X}{ Pace, R. K., \& Barry, R. (1997). "Sparse Spatial Autoregressions." Statistics \& Probability Letters, 33(3), 291-297}