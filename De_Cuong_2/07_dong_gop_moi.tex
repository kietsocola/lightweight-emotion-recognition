\section{Những đóng góp mới của đề tài} % Section 7

Đề tài không chỉ tiếp cận bài toán nhận diện biểu cảm khuôn mặt theo hướng tối ưu tài nguyên mà còn đề xuất cách tiếp cận hiệu quả trong điều kiện ánh sáng yếu – một thách thức phổ biến trong ứng dụng thực tiễn. Những đóng góp mới của đề tài bao gồm:

\begin{itemize}
    \item \textbf{Đề xuất kỹ thuật tăng cường dữ liệu thích ứng theo mức độ sáng:} Khác với các phương pháp tăng cường cố định truyền thống, nhóm đã thiết kế cơ chế tăng cường ảnh linh hoạt dựa trên đặc trưng ánh sáng đầu vào (trung bình độ sáng, histogram, v.v.). Kỹ thuật này giúp cải thiện khả năng học và tổng quát hóa của mô hình trong các tình huống ánh sáng không đồng đều.

    \item \textbf{Ứng dụng mô hình CNN nhẹ (MobileNetV3) trong điều kiện ánh sáng yếu:} Đề tài chứng minh rằng, với phương pháp xử lý dữ liệu phù hợp, các mô hình CNN nhẹ hoàn toàn có thể đạt hiệu suất tốt mà vẫn đảm bảo tốc độ suy luận nhanh – mở ra khả năng triển khai thực tế trong các hệ thống giám sát, thiết bị IoT, hoặc môi trường có giới hạn tài nguyên.

    \item \textbf{Tạo tập dữ liệu ánh sáng yếu có tính đại diện cao:} Nhóm đã xây dựng tập dữ liệu mô phỏng ánh sáng yếu từ FER-2013 với nhiều cấp độ sáng khác nhau, phục vụ cho huấn luyện và đánh giá mô hình. Đây có thể là tài nguyên tham khảo hữu ích cho các nghiên cứu liên quan trong tương lai.

    \item \textbf{Thực nghiệm so sánh có định hướng rõ ràng:} Đề tài không chỉ triển khai mô hình mà còn thực hiện đánh giá định lượng giữa các phương pháp có và không có tăng cường thích ứng, giúp làm rõ hiệu quả thực sự của hướng tiếp cận đề xuất.

\end{itemize}

Những đóng góp trên, tuy ở mức ứng dụng, nhưng có ý nghĩa thực tiễn cao và có thể trở thành tiền đề cho các nghiên cứu chuyên sâu hơn trong lĩnh vực nhận diện biểu cảm trong điều kiện ánh sáng không lý tưởng.