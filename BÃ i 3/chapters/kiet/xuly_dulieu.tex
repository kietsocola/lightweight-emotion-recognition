\textbf{Có.}. Tác giả đã thực hiện nhiều bước tiền xử lý dữ liệu để đảm bảo chất lượng đầu vào cho mô hình, bao gồm:

\textbf{Xử lý dữ liệu bị thiếu}
\begin{itemize}
    \item Các cột có quá nhiều giá trị bị thiếu (>40\%) như \textit{PoolQC, MiscFeature, Alley, Fence, FireplaceQu} bị loại bỏ.
    \item Các cột còn lại được điền giá trị phù hợp, ví dụ:
    \begin{itemize}
        \item \textit{LotFrontage} được điền bằng giá trị trung vị của khu vực (\textit{Neighborhood}).
        \item Các đặc trưng liên quan đến tầng hầm được điền bằng ``None'' nếu không có tầng hầm.
    \end{itemize}
\end{itemize}

\textbf{Xử lý outlier (giá trị ngoại lai)}
\begin{itemize}
    \item Hai giá trị có diện tích sàn lớn (\textit{GrLivArea}) nhưng giá bán thấp bị loại bỏ vì chúng làm méo mô hình.
\end{itemize}

\textbf{Biến đổi dữ liệu}
\begin{itemize}
    \item Áp dụng \textbf{log transformation} lên giá nhà (\textit{SalePrice}) để biến đổi phân phối lệch phải thành phân phối chuẩn hơn.
\end{itemize}

\textbf{Xử lý multicollinearity (đa cộng tuyến)}
\begin{itemize}
    \item Một số biến có tương quan cao bị loại bỏ để tránh dư thừa, ví dụ:
    \begin{itemize}
        \item \textit{GarageCars} giữ lại, còn \textit{GarageArea} bị loại bỏ.
        \item \textit{TotalBsmtSF} giữ lại, còn \textit{1stFlrSF} bị loại bỏ.
    \end{itemize}
\end{itemize}

\textbf{Mã hóa dữ liệu dạng danh mục (categorical encoding)}
\begin{itemize}
    \item Dữ liệu dạng danh mục được mã hóa thành dạng số để máy học có thể xử lý.
\end{itemize}

\textbf{Chuyển đổi dữ liệu lệch (skewed features)}
\begin{itemize}
    \item Áp dụng \textbf{Box-Cox transformation} cho các biến có độ lệch lớn (>0.75) để giúp mô hình hoạt động ổn định hơn.
\end{itemize}
