\section{Kết luận và hướng phát triển}

\subsection{Tóm tắt kết quả nghiên cứu}

Nghiên cứu đã xây dựng thành công một hệ thống nhận diện biểu cảm khuôn mặt trong điều kiện ánh sáng yếu, dựa trên mô hình CNN nhẹ MobileNetV3 kết hợp với pipeline tăng cường dữ liệu thích ứng. Phương pháp xử lý ảnh thích ứng dựa trên phân tích đặc trưng độ sáng của ảnh đầu vào đã góp phần nâng cao chất lượng hình ảnh, từ đó cải thiện đáng kể hiệu suất nhận diện. Kết quả thực nghiệm trên tập dữ liệu FER-2013 cho thấy mô hình đạt độ chính xác 61.55% trong điều kiện ánh sáng yếu với kỹ thuật tăng cường dữ liệu thích ứng, gần tương đương với hiệu suất trên ảnh gốc. Đồng thời, so sánh với mô hình ResNet18 cho thấy MobileNetV3 có ưu thế về kích thước và thời gian suy luận, phù hợp cho các ứng dụng yêu cầu tài nguyên thấp.

Nghiên cứu cũng chỉ ra một số hạn chế, bao gồm độ chính xác tổng thể còn khiêm tốn và hiện tượng nhầm lẫn giữa các biểu cảm có đặc trưng tương đồng, nguyên nhân chủ yếu do mất cân bằng dữ liệu và sự gần giống của các biểu cảm.

\subsection{Câu hỏi nghiên cứu}

Nghiên cứu tập trung trả lời các câu hỏi sau:

\begin{enumerate}
\item Làm thế nào để thiết kế một pipeline tăng cường dữ liệu thích ứng, hiệu quả trong việc cải thiện chất lượng ảnh ánh sáng yếu phục vụ nhận diện biểu cảm khuôn mặt?
\item Mô hình MobileNetV3 có thể đạt được độ chính xác tương đương hoặc vượt trội so với các kỹ thuật tăng cường dữ liệu cố định trong điều kiện ánh sáng yếu hay không?
\item Các kỹ thuật tăng cường dữ liệu thích ứng ảnh hưởng như thế nào đến hiệu suất của mô hình CNN nhẹ trong nhiệm vụ nhận diện biểu cảm khuôn mặt?
\end{enumerate}

Thông qua quá trình nghiên cứu và thử nghiệm, các câu hỏi trên được giải đáp như sau:

\begin{itemize}
\item Pipeline tăng cường dữ liệu thích ứng được thiết kế dựa trên phân tích độ sáng trung bình và độ lệch chuẩn của ảnh đầu vào, giúp điều chỉnh linh hoạt các phép biến đổi ảnh phù hợp với đặc điểm từng ảnh. Giải pháp này đã chứng minh hiệu quả trong việc cải thiện chất lượng ảnh ánh sáng yếu, từ đó nâng cao độ chính xác nhận diện.
\item Mô hình MobileNetV3 khi kết hợp với kỹ thuật tăng cường dữ liệu thích ứng đạt hiệu suất gần bằng với mô hình trên ảnh gốc và vượt trội so với kỹ thuật tăng cường dữ liệu cố định, chứng tỏ khả năng thích nghi tốt với các điều kiện ánh sáng khác nhau.
\item Kỹ thuật tăng cường dữ liệu thích ứng giúp giảm thiểu hiện tượng mất chi tiết trên khuôn mặt trong môi trường thiếu sáng, từ đó cải thiện hiệu suất tổng thể của mô hình CNN nhẹ, mặc dù vẫn tồn tại một số hạn chế liên quan đến sự nhầm lẫn giữa các biểu cảm có đặc trưng tương đồng.
\end{itemize}

\subsection{Hướng phát triển}

Trên cơ sở kết quả đạt được và các hạn chế còn tồn tại, nghiên cứu đề xuất các hướng phát triển tiếp theo như sau:

\begin{itemize}
\item Mở rộng tập dữ liệu ánh sáng yếu đa dạng hơn về biểu cảm, độ tuổi, giới tính và điều kiện môi trường nhằm nâng cao khả năng tổng quát hóa của mô hình.
\item Nghiên cứu và áp dụng các kỹ thuật tiền xử lý ảnh hiệu quả hơn, đảm bảo tốc độ xử lý nhanh và phù hợp với thiết bị nhúng.
\item Tối ưu hóa mô hình CNN nhẹ thông qua các phương pháp lượng tử hóa, cắt tỉa hoặc kiến trúc mới nhằm giảm kích thước và thời gian suy luận, đồng thời duy trì hoặc nâng cao độ chính xác.
\item Phát triển các kỹ thuật học đặc trưng nâng cao hoặc mô hình đa nhiệm để cải thiện khả năng phân biệt các biểu cảm tương đồng.
\item Triển khai và đánh giá hệ thống trên các thiết bị thực tế trong môi trường ánh sáng yếu nhằm xác định tính khả thi và hiệu quả ứng dụng thực tiễn.
\end{itemize}

Nghiên cứu kỳ vọng sẽ góp phần thúc đẩy phát triển các giải pháp nhận diện biểu cảm khuôn mặt hiệu quả và khả thi trong điều kiện ánh sáng yếu, phục vụ các ứng dụng đa dạng trong lĩnh vực giám sát an ninh, tương tác người - máy và chăm sóc sức khỏe tâm lý.