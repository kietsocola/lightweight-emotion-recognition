\section{Lý do chọn đề tài}

Trong thời đại công nghệ số và trí tuệ nhân tạo phát triển mạnh mẽ, các hệ thống nhận diện biểu cảm khuôn mặt (Facial Expression Recognition - FER) ngày càng đóng vai trò quan trọng trong nhiều lĩnh vực như giám sát an ninh, giáo dục thông minh, chăm sóc sức khỏe, và tương tác người - máy. Tuy nhiên, một trong những thách thức lớn mà các hệ thống FER hiện nay gặp phải là điều kiện ánh sáng không ổn định, đặc biệt là ánh sáng yếu – tình huống phổ biến trong môi trường thực tế như ban đêm, nhà thông minh hoặc thiết bị giám sát có cấu hình thấp.

Nhiều nghiên cứu hiện tại đã áp dụng các mô hình học sâu như CNN, GAN hay các phương pháp tiền xử lý hình ảnh phức tạp để cải thiện hiệu suất trong điều kiện ánh sáng yếu. Tuy nhiên, phần lớn các phương pháp này đòi hỏi tài nguyên tính toán cao, khó triển khai trên thiết bị thực tế như camera nhúng, điện thoại cấu hình thấp hoặc hệ thống giám sát nhỏ gọn.

Xuất phát từ thực tiễn đó, nhóm chúng em lựa chọn nghiên cứu và xây dựng một hệ thống FER đơn giản, hiệu quả và phù hợp với thiết bị hạn chế tài nguyên. Đề tài tập trung vào việc kết hợp mô hình CNN nhẹ (MobileNetV3) với kỹ thuật tăng cường dữ liệu thích ứng theo mức độ sáng nhằm nâng cao khả năng nhận diện biểu cảm trong điều kiện ánh sáng yếu. Với hướng tiếp cận này, đề tài kỳ vọng có thể góp phần vào việc phát triển các hệ thống nhận diện thông minh có tính khả thi cao trong ứng dụng thực tiễn.